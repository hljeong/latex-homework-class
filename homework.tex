\documentclass{homework}

\class{MATH 131AH (Real Analysis)}
\num{6}
\name{Lucas Jeong}

\subproblemLabelStyle{arabic}
\solutionColor{red}

\begin{document}

\problem An ultrametric on $X$ is a metric $\rho$ on $X$ such that 
\[
  \forall \, x, y, z \in X: \rho(x, y) \leq
    \max \left \{ \rho(x, z), \rho(y, z) \right \}.
\] 
Prove that, in this metric, every open ball 
  $B(x, r) := \left \{ y \in X ~ : ~ \rho(x, y) < r \right \}$ is closed 
  and every closed ball $B'(x, r) := \left \{ y \in X ~ : ~ 
  \rho(x, y) \leq r \right \}$ is open.
Determine the topological boundary $\partial B(x, r)$ of $B(x, r)$.

To give an example of such a setting, let $X := \left \{ 0, 1 \right \}^
{\mathbb N}$.
Prove that then $\rho : X \times X \to \mathbb R$ defined for 
$\sigma, \sigma' \in X$ with $\sigma \neq \sigma'$ by 
\[
  \rho(\sigma, \sigma') := %
    2^{-\inf\left \{ k \in \mathbb N ~ : ~ \sigma(k) \neq \sigma'(k) \right \}}
\]
(and by $\rho(\sigma, \sigma) := 0$) is an ultrametric.

  \solution Note that for all $y \in X \setminus B(x, r)$ there is 
    $\rho(x, y) \geq r$.
  Then for all $z \in B(y, r)$ there is $\rho(x, y) \leq 
    \max\left \{ \rho(x, z), \rho(z, y) \right \}$.
  Since $\rho(z, y) < r \leq \rho(x, y)$, there must be 
    $\rho(x, z) \geq \rho(x, y) \geq r$.
  Then $z \in X \setminus B(x, r)$.
  Then $X \setminus B(x, r)$ is open and $B(x, r)$ is closed.
    
  Note that $\partial B(x, r) = \left \{ b \in X ~ : ~ 
    \forall r > 0: B(x, r) \cap B(b, r) \neq  \varnothing \wedge 
    B(b, r) \setminus B(x, r) \neq \varnothing \right \}$. \\ 
  Suppose $b \in \partial B(x, r)$.
  Then there exists $y \in X$ such that $\rho(x, y) < r$ and $\rho(b, y) < r$.
  There also exists $z \in X$ such that $\rho(z, b) < r$ and 
    $\rho(z, x) \leq r$.
  However, there is $\rho(z, x) \leq 
    \max\left \{ \rho(z, b), \rho(b, x) \right \} = 
    \max\left \{ \rho(z, b), \rho(b, y), \rho(y, x) \right \} < r$, 
    resulting in a contradiction.
  Then $\partial B(x, r) = \varnothing$.
  
  Note that exponential functions are positive over $\mathbb R$.
  Then $\rho(\sigma, \sigma') \geq 0$ and equality holds only when 
    $\sigma = \sigma'$.

  Since inequality is symmetric ($\sigma(k) \neq \sigma'(k) \Leftrightarrow 
    \sigma'(k) \neq \sigma(k)$), $\rho$ is symmetric.

  Suppose for the sake of contradiction that for some $\sigma, \sigma', 
    \sigma'' \in X$, $\rho(\sigma, \sigma'') > 
    \max\left \{ \rho(\sigma, \sigma'), \rho(\sigma', \sigma'') \right \}$.
  Let $i = \inf\left \{ k \in \mathbb N ~ : ~ \sigma(k) \neq 
    \sigma''(k) \right \}$, $j = \inf\left \{ k \in \mathbb N ~ : ~ 
    \sigma(k) \neq \sigma'(k) \right \}$, and $l = 
    \inf\left \{ k \in \mathbb N ~ : ~ \sigma'(k) \neq \sigma''(k) \right \}$.
  Then $i < j$ and $i < k$.
  Then $\sigma(k) = \sigma'(k) = \sigma''(k)$ for all $k = 1, \dots, 
    \min\left \{ j, l \right \}$ and $i$ is not the greatest lower bound of 
    $\left \{ k \in \mathbb N ~ : ~ \sigma(k) \neq \sigma''(k) \right \}$, 
   resulting in a contradiction.
 Then $\rho(\sigma, \sigma'') \leq \max\left \{ \rho(\sigma, \sigma'), 
  \rho(\sigma', \sigma'') \right \} \leq \rho(\sigma, \sigma') + 
  \rho(\sigma', \sigma'')$.
  
  Then $\rho$ is an ultrametric on $X$.


\problem[Baby Rudin p.43 exercise 9] Let $E^\circ$ denote the set of all 
  interior points of a set $E$. 

\begin{subproblems}
  \subproblem Prove that $E^\circ$ is always open.

    \solution $E^\circ$ is open by definition: 
    for all $x \in E^\circ$ there exists $r > 0$ such that 
    $B(x, r) \subseteq A$.

  \subproblem Prove that $E$ is open if and only if $E^\circ = E$.
  
    \solution If $E = E^\circ$ then $E$ is clearly open.

    If $E$ is open then for all $x \in E$ there exists $r > 0$ such that 
      $B(x, r) \subseteq E$.
    Then $x \in E^\circ$.
    Then $E \subseteq E^\circ$.
    For all $x \in E^\circ$, there exists $r > 0$ such that 
      $B(x, r) \subseteq E$.
    Then $x \in E$.
    Then $E^\circ \subseteq E$.
    Then $E = E^\circ$.

  \subproblem If $G \subseteq E$ and $G$ is open, prove that 
  $G \subseteq E^\circ$.
  
    \solution Suppose $G$ is open and $G \subseteq E$.
    Then for all $x \in G$ there exists $r > 0$ such that 
      $B(x, r) \subseteq G \subseteq E$.
    Then $x \in E^\circ$.

  \subproblem Prove that the complement of $E^\circ$ is the closure 
  of the complement of $E$.
  
    \solution Note that since for all open subsets $G \subseteq E$ there 
      is $G \subseteq E^\circ$, 
      $E^\circ = \bigcup \left \{ G \subseteq E ~ : ~ 
      G \text{ is open} \right \}$.
    Then, by De Morgan's law, there is $X \setminus E^\circ = \bigcap 
      \left \{ X \setminus G ~ : ~ 
      G \subseteq E \wedge G \text{ is open} \right \} = \\
      \bigcap \left \{ F \subseteq X ~ : ~ 
      X \setminus E \subseteq F \wedge F \text{ is closed} \right \} = 
      \overline{X \setminus E}$.

  \subproblem Do $E$ and $\overline E$ always have the same interiors?
  
    \solution No.
    Let $X = \mathbb R$ and $E = \mathbb Q$.
    Note that $E^\circ = \varnothing$ while $\overline E = \mathbb R$ and 
      $\overline E^\circ = \mathbb R$.

  \subproblem Do $E$ and $E^\circ$ always have the same closures?
  
    \solution No.
    Let $X = \mathbb R$ and $E = \mathbb Q$.
    Note that $\overline E = \mathbb R$ while $E^\circ = \varnothing$ and
      $\overline{E^\circ} = \varnothing$.
\end{subproblems}

\end{document}